#                            List.cpp中测试代码的设计思路

#### 1. **初始化列表构造链表**

- **目标**:测试使用初始化列表创建链表的功能,验证链表的初始构造是否正确。

- **设计**:

  - 使用初始化列表 `{1, 2, 3, 4, 5}` 创建一个链表 `link1`。
  - 调用 `printList()` 输出链表的内容,确保每个节点的值按顺序被正确插入到链表中。

- **代码**:

  ```
  SingleLinkedList<int> link1 {1, 2, 3, 4, 5};
  std::cout << "Initial List (link1): ";
  link1.printList();
  ```

- **预期输出**:

  ``` cpp
  Initial List (link1): 1 2 3 4 5
  ```

  

- **原因**:链表的初始化是构造链表的第一步,必须确保使用初始化列表时,链表按顺序存储所有元素,且 `head` 和 `currentPos` 被正确设置。

------

#### 2. **复制构造函数**

- **目标**:测试链表的复制构造函数,验证是否可以正确深拷贝链表。

- **设计**:

  - 通过 `SingleLinkedList<int> link2(link1);` 复制构造 `link1`。
  - 调用 `printList()` 检查 `link2` 是否与 `link1` 中的元素一致,确保深拷贝正确。

- **代码**:

  ```
  SingleLinkedList<int> link2(link1);
  std::cout << "Copied List (link2): ";
  link2.printList();
  ```

- **预期输出**:

  ``` cpp
  Copied List (link2): 1 2 3 4 5
  ```

  

- **原因**:复制构造函数应确保链表中每个节点的值都被正确复制,并且 `link2` 的元素与 `link1` 完全相同且不共享同一块内存。

------

#### 3. **赋值操作符**

- **目标**:测试赋值操作符 `operator=`,确保链表能够正确进行赋值。

- **设计**:

  - 使用 `link3 = link1;` 赋值 `link1` 给一个新链表 `link3`。
  - 调用 `printList()` 输出 `link3` 的内容,验证赋值是否成功。

- **代码**:

  ```
  SingleLinkedList<int> link3;
  link3 = link1;
  std::cout << "Assigned List (link3): ";
  link3.printList();
  ```

- **预期输出**:

- ``` cpp
  Assigned List (link3): 1 2 3 4 5
  ```

- 

- **原因**:赋值操作符需要处理深拷贝情况,因此赋值后的 `link3` 必须与 `link1` 相同,但独立。

------

#### 4. **插入元素**

- **目标**:测试 `insert()` 函数,验证插入功能是否正确,尤其是插入元素后链表的连接是否保持完整。

- **设计**:

  - 使用 `link1.insert(10);` 向链表中插入元素 `10`,在 `currentPos` 之后插入。
  - 调用 `printList()` 检查插入后的链表是否正确更新。

- **代码**:

  ```
  std::cout << "Insert 10 after current element in link1." << std::endl;
  link1.insert(10);
  link1.printList();
  ```

- **预期输出**:

  ``` cpp
  Insert 10 after current element in link1.
  1  10  2  3  4  5
  ```

  

- **原因**:插入操作是链表常见的操作之一,必须验证插入后链表的顺序和指针连接是否保持一致。

------

#### 5. **查找元素**

- **目标**:测试 `find()` 函数,验证在链表中查找特定元素的功能。

- **设计**:

  - 先使用 `link1.find(3);` 查找元素 `3`。
  - 调用 `getCurrentVal()` 输出查找到的元素,验证查找是否成功。

- **代码**:

  ```
  if (link1.find(3)) {
      std::cout << "Element 3 found in link1. Current value: " << link1.getCurrentVal() << std::endl;
  }
  ```

- **预期输出**:

  ``` cpp
  Element 3 found in link1. Current value: 3
  ```

  

- **原因**:查找操作能够让用户在链表中定位到特定元素,确保链表的查找功能正常工作,并更新 `currentPos`。

------

#### 6. **获取和设置当前元素值**

- **目标**:测试 `getCurrentVal()` 和 `setCurrentVal()`,验证链表中获取和修改当前节点值的功能。

- **设计**:

  - 使用 `link1.find(10)` 查找到元素 `10`。
  - 调用 `setCurrentVal(20)` 修改当前节点的值为 `20`。
  - 再次使用 `printList()` 检查链表是否更新成功。

- **代码**:

  ```
  if (link1.find(10)) {
      std::cout << "Set current value to 20 (was " << link1.getCurrentVal() << ")" << std::endl;
      link1.setCurrentVal(20);
      link1.printList();
  }
  ```

- **预期输出**:

  ``` cpp
  Set current value to 20 (was 10)
  1  20  2  3  4  5
  ```

  

- **原因**:此操作验证链表中节点数据的访问和修改功能,确保链表中元素的值能正确获取和设置。

------

#### 7. **删除最后一个元素**

- **目标**:测试 `remove()` 函数,验证删除链表中最后一个节点的功能。

- **设计**:

  - 使用 `link1.remove()` 删除链表中的最后一个节点。
  - 调用 `printList()` 检查链表更新后的状态。

- **代码**:

  ```
  std::cout << "Remove last element in link1." << std::endl;
  link1.remove();
  link1.printList();
  ```

- **预期输出**:

  ``` cpp
  Remove last element in link1.
  1  20  2  3  4
  ```

  

- **原因**:删除操作需要验证链表末端的指针是否被正确处理,以及是否释放了最后一个节点的内存。

------

#### 8. **检查链表是否为空**

- **目标**:测试 `isEmpty()` 函数,验证链表是否为空的检查功能。

- **设计**:

  - 调用 `link1.isEmpty()` 检查链表是否为空,输出结果。

- **代码**:

  ```
  std::cout<< "Is link1 empty? " << (link1.isEmpty() ? "Yes" : "No") << std::endl;
  ```

- **预期输出**:

  ``` cpp
  Is link1 empty? No
  ```

  

- **原因**:验证链表是否为空是常见的链表状态检查,需要确保函数返回的结果与链表实际内容一致。

------

#### 9. **获取链表大小**

- **目标**:测试 `getSize()` 函数,验证获取链表大小的功能。

- **设计**:

  - 调用 `link1.getSize()` 获取链表当前的大小并输出。

- **代码**:

  ```
  std::cout << "Size of link1: " << link1.getSize() << std::endl;
  ```

- **预期输出**:

  ``` cpp
  Size of link1: 5
  ```

  

- **原因**:链表大小需要随着插入和删除操作的变化而正确更新,因此该功能确保链表的节点数量被正确计算。

------

#### 10. **清空链表**

- **目标**:测试 `emptyList()` 函数,验证清空整个链表的功能。

- **设计**:

  - 使用 `link1.emptyList()` 清空链表。
  - 调用 `isEmpty()` 检查链表是否已清空,并使用 `getSize()` 验证链表大小为 `0`。

- **代码**:

  ```
  std::cout << "Empty link1." << std::endl;
  link1.emptyList();
  std::cout << "Is link1 empty after emptying? " << (link1.isEmpty() ? "Yes" : "No") << std::endl;
  std::cout << "Size of link1 after emptying: " << link1.getSize() << std::endl;
  ```

- **预期输出**:

- ``` cpp
  Empty link1.
  Is link1 empty after emptying? Yes
  Size of link1 after emptying: 0
  ```

- 

  **原因**:清空链表涉及释放所有节点内存,确保链表在清空后回到初始状态,且所有节点正确删除。







