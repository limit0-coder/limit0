###                                                         Remove* 函数的实现

​        remove函数的目的是删除树中的一个元素,并确保删除操作后,二叉搜索树(BST)的结构和AVL树的平衡性质不受破坏。具体操作包括以下几步:

##### 1.递归查找目标元素:如果当前节点为空,直接返回;否则,按大小关系递归地查找。

##### 2.匹配到目标元素后删除:

(1)  如果该节点有两个子节点,找到其右子树中的最小节点,并将其值替换到当前节点。
(2)  如果该节点只有一个子节点或没有子节点,将其删除并用其子节点(或nullptr)替代。

##### 3.删除后平衡树:每次删除操作后,需要通过balance函数确保树的平衡性。

##### 具体实现

##### 1. **递归查找目标元素**

- 初始调用`remove`时,会传入目标元素`x`和树的根节点`root`。

- 如果当前节点为空(`t == nullptr`),表示树中没有该元素,直接返回。

- 如果目标元素小于当前节点的元素(`x < t->element`),则递归地查找左子树;如果大于当前节点的元素(`x > t->element`),则递归地查找右子树。

  ##### 2.删除目标节点

  当匹配到目标元素时,有三种不同的情况:

  (1) 当前节点有两个子节点

  如果当前节点有两个子节点,需要从右子树中找到最小节点,并用该最小节点的元素替代当前节点的元素。这是因为,按照二叉搜索树的性质,右子树的最小节点一定比当前节点大,并且不会比当前节点的右子树大,在此过程中使用detachMin(t->right)和balance(t)函数

  (2) 当前节点只有一个子节点或没有子节点

  如果当前节点只有一个子节点或没有子节点,直接将当前节点的子树替代当前节点。这里使用三元运算符,判断左子树是否为空,若不为空,左子树将替代当前节点;如果左子树为空,则使用右子树替代当前节点,使用delete oldNode删除原节点,释放内存。

  ##### **AVL树的平衡操作**

  删除操作后,`balance`函数确保树的平衡。它会通过计算当前节点的平衡因子(左右子树的高度差)来判断是否需要进行旋转。

  **`updateHeight(t)`**:更新当前节点的高度。

  **`getBalance(t)`**:计算当前节点的平衡因子(左子树高度 - 右子树高度)。

  根据平衡因子的值:

    如果平衡因子大于1(左子树过高),可能需要进行右旋;如果左子树的右子树过高,则先进行左旋。

    如果平衡因子小于-1(右子树过高),可能需要进行左旋;如果右子树的左子树过高,则先进行右旋。

  ##### 总结

  1.`remove`函数的主要作用是删除指定元素,并根据节点的结构进行不同的删除操作。删除后,树的平衡性通过`balance`函数进行修正。

  2.在删除具有两个子节点的节点时,通过替换右子树的最小节点来保持二叉搜索树的性质。

  3.AVL树的平衡操作通过旋转(左旋和右旋)来恢复树的平衡。

  

  

